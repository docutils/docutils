\documentclass[a4paper]{article}
% generated by Docutils <https://docutils.sourceforge.io/>
% rubber: set program xelatex
\usepackage{fontspec}
% \defaultfontfeatures{Scale=MatchLowercase}
% straight double quotes (defined in T1 but missing in TU):
\ifdefined \UnicodeEncodingName
  \DeclareTextCommand{\textquotedbl}{\UnicodeEncodingName}{%
    {\addfontfeatures{RawFeature=-tlig,Mapping=}\char34}}%
\fi
\usepackage{alltt}
\usepackage{amsmath}
\usepackage{polyglossia}
\setdefaultlanguage{english}
\setotherlanguages{english,french,german}
\usepackage{color}
\usepackage{float} % extended float configuration
\floatplacement{figure}{H} % place figures here definitely
\usepackage{graphicx}
\usepackage{multirow}
\setcounter{secnumdepth}{0}
\usepackage{longtable,ltcaption,array}
\setlength{\extrarowheight}{2pt}
\newlength{\DUtablewidth} % internal use in tables
\usepackage{tabularx}

%%% Custom LaTeX preamble
% Linux Libertine (free, wide coverage, not only for Linux)
\setmainfont{Linux Libertine O}
\setsansfont{Linux Biolinum O}
\setmonofont[HyphenChar=None,Scale=MatchLowercase]{DejaVu Sans Mono}

%%% User specified packages and stylesheets

%%% Fallback definitions for Docutils-specific commands
\usepackage{docutils}[2024-09-24]
\newcounter{enumv}

\DUprovidelength{\pdfpxdimen}{1bp}

\providecommand*{\DUCLASScontents}{%
  \renewenvironment{itemize}%
    {\begin{list}{}{\setlength{\partopsep}{0pt}
                    \setlength{\parsep}{0pt}}
                   }%
    {\end{list}}%
}
% hyperlinks:
\ifdefined\hypersetup
\else
  \usepackage[hyperfootnotes=false,
              colorlinks=true,linkcolor=blue,urlcolor=blue]{hyperref}
  \usepackage{bookmark}
  \urlstyle{same} % normal text font (alternatives: tt, rm, sf)
\fi
\hypersetup{
  pdftitle={reStructuredText Test Document},
  pdfkeywords={reStructuredText, test, parser},
  pdfinfo={description={A test document, containing at least one example of each reStructuredText construct.}},
  pdfauthor={David Goodger; Me; Myself; I}
}

\title{reStructuredText Test Document%
  \label{restructuredtext-test-document}%
  \label{doctitle}%
  \\%
  \DUdocumentsubtitle{Examples of Syntax Constructs}%
  \label{examples-of-syntax-constructs}%
  \label{subtitle}}
\author{}
\date{}

%%% Body
\begin{document}
\maketitle

% Docinfo
\begin{center}
\begin{tabularx}{\DUdocinfowidth}{lX}
\textbf{Author}: & David Goodger \\
\textbf{Address}: & {\raggedright
123 Example Street\\
Example, EX  Canada\\
A1B 2C3} \\
\textbf{Contact}: & \href{mailto:goodger@python.org}{goodger@python.org} \\
\textbf{Authors}: & Me, Myself, I \\
\textbf{Organization}: & humankind \\
\textbf{Date}: & Now, or yesterday.  Or maybe even \emph{before} yesterday. \\
\textbf{Status}: & This is a “work in progress” \\
\textbf{Revision}: & is managed by a version control system. \\
\textbf{Version}: & 1 \\
\textbf{Copyright}: &
  This document has been placed in the public domain. You
may do with it as you wish. You may copy, modify,
redistribute, reattribute, sell, buy, rent, lease,
destroy, or improve it, quote it at length, excerpt,
incorporate, collate, fold, staple, or mutilate it, or do
anything else to it that your or anyone else’s heart
desires. \\
\textbf{field name}: &
This is a “generic bibliographic field”.
\\
\textbf{field name “2”}: &
Generic bibliographic fields may contain multiple body elements.

Like this.
\\
\end{tabularx}
\end{center}

\begin{DUclass}{dedication}
\begin{quote}
\DUtitle{Dedication}

For Docutils users \& co-developers.
\end{quote}
\end{DUclass}

\begin{DUclass}{abstract}
\begin{quote}
\DUtitle{Abstract}

This is a test document, containing at least one example of each
reStructuredText construct.
\end{quote}
\end{DUclass}

% This is a comment. Note how any initial comments are moved by
% transforms to after the document title, subtitle, and docinfo.

% Above is the document title, and below is the subtitle.
% They are transformed from section titles after parsing.

% bibliographic fields (which also require a transform):

\pagebreak[4] % start ToC on new page

\phantomsection\label{table-of-contents}
\pdfbookmark[1]{Table of Contents}{table-of-contents}

\begin{DUclass}{contents}

\DUtitle{Table of Contents}

\begin{DUclass}{auto-toc}
\begin{itemize}
\item \hyperref[structural-elements]{1   Structural Elements}

\begin{DUclass}{auto-toc}
\begin{itemize}
\item \hyperref[section-title]{1.1   Section Title}

\item \hyperref[empty-section]{1.2   Empty Section}

\item \hyperref[transitions]{1.3   Transitions}
\end{itemize}
\end{DUclass}

\item \hyperref[body-elements]{2   Body Elements}

\begin{DUclass}{auto-toc}
\begin{itemize}
\item \hyperref[paragraphs]{2.1   Paragraphs}

\begin{DUclass}{auto-toc}
\begin{itemize}
\item \hyperref[inline-markup]{2.1.1   Inline Markup}
\end{itemize}
\end{DUclass}

\item \hyperref[bullet-lists]{2.2   Bullet Lists}

\item \hyperref[enumerated-lists]{2.3   Enumerated Lists}

\item \hyperref[definition-lists]{2.4   Definition Lists}

\item \hyperref[field-lists]{2.5   Field Lists}

\item \hyperref[option-lists]{2.6   Option Lists}

\item \hyperref[literal-blocks]{2.7   Literal Blocks}

\item \hyperref[line-blocks]{2.8   Line Blocks}

\item \hyperref[block-quotes]{2.9   Block Quotes}

\item \hyperref[doctest-blocks]{2.10   Doctest Blocks}

\item \hyperref[footnotes]{2.11   Footnotes}

\item \hyperref[citations]{2.12   Citations}

\item \hyperref[targets]{2.13   Targets}

\begin{DUclass}{auto-toc}
\begin{itemize}
\item \hyperref[duplicate-target-names]{2.13.1   Duplicate Target Names}

\item \hyperref[duplicate-target-names-1]{2.13.2   Duplicate Target Names}
\end{itemize}
\end{DUclass}

\item \hyperref[directives]{2.14   Directives}

\begin{DUclass}{auto-toc}
\begin{itemize}
\item \hyperref[document-parts]{2.14.1   Document Parts}

\item \hyperref[images-and-figures]{2.14.2   Images and Figures}

\item \hyperref[tables]{2.14.3   Tables}

\item \hyperref[admonitions]{2.14.4   Admonitions}

\item \hyperref[topics-sidebars-and-rubrics]{2.14.5   Topics, Sidebars, and Rubrics}

\item \hyperref[target-footnotes]{2.14.6   Target Footnotes}

\item \hyperref[replacement-text]{2.14.7   Replacement Text}

\item \hyperref[compound-paragraph]{2.14.8   Compound Paragraph}

\item \hyperref[parsed-literal-blocks]{2.14.9   Parsed Literal Blocks}

\item \hyperref[code]{2.14.10   Code}

\item \hyperref[meta]{2.14.11   Meta}
\end{itemize}
\end{DUclass}

\item \hyperref[substitution-definitions]{2.15   Substitution Definitions}

\item \hyperref[comments]{2.16   Comments}

\item \hyperref[raw-text]{2.17   Raw text}

\item \hyperref[container]{2.18   Container}

\item \hyperref[colspanning-tables]{2.19   Colspanning tables}

\item \hyperref[rowspanning-tables]{2.20   Rowspanning tables}

\item \hyperref[list-tables]{2.21   List Tables}

\item \hyperref[custom-roles]{2.22   Custom Roles}

\item \hyperref[mathematics]{2.23   Mathematics}
\end{itemize}
\end{DUclass}

\item \hyperref[tests-for-the-latex-writer]{3   Tests for the LaTeX writer}

\begin{DUclass}{auto-toc}
\begin{itemize}
\item \hyperref[custom-roles-in-latex]{3.1   Custom Roles in LaTeX}

\item \hyperref[class-handling]{3.2   class handling}
\end{itemize}
\end{DUclass}

\item \hyperref[tests-for-the-xetex-writer]{4   Tests for the XeTeX writer}

\item \hyperref[error-handling]{5   Error Handling}
\end{itemize}
\end{DUclass}
\end{DUclass}


\section{1   Structural Elements%
  \label{structural-elements}%
}


\subsection{1.1   Section Title%
  \label{section-title}%
}
\subsubsection*{Section Subtitle}

Lone subsections are converted to a section subtitle by a transform
activated with the \texttt{--section-subtitles} command line option or the
\texttt{sectsubtitle-xform} configuration value.


\subsection{1.2   Empty Section%
  \label{empty-section}%
}


\subsection{1.3   Transitions%
  \label{transitions}%
}

Here’s a transition:

%___________________________________________________________________________
\DUtransition

It divides the section.  Transitions may also occur between sections:

%___________________________________________________________________________
\DUtransition


\section{2   Body Elements%
  \label{body-elements}%
}


\subsection{2.1   Paragraphs%
  \label{paragraphs}%
}

A paragraph.


\subsubsection{2.1.1   Inline Markup%
  \label{inline-markup}%
}

Paragraphs contain text and may contain inline markup: \emph{emphasis},
\textbf{strong emphasis}, \texttt{inline literals}, standalone hyperlinks
(\url{http://www.python.org}), external hyperlinks (\href{http://www.python.org/}{Python}\DUfootnotemark{footnote-reference-10}{footnote-6}{5}), internal
cross-references (\hyperref[example]{example}), external hyperlinks with embedded URIs
(\href{http://www.python.org}{Python web site}), \href{http://www.python.org/}{anonymous hyperlink
references}\DUfootnotemark{footnote-reference-17}{footnote-6}{5} (\href{https://docutils.sourceforge.io/}{a second reference}\DUfootnotemark{footnote-reference-18}{footnote-10}{9}), footnote references (manually
numbered\DUfootnotemark{footnote-reference-1}{footnote-1}{1}, anonymous auto-numbered\DUfootnotemark{footnote-reference-2}{footnote-2}{3}, labeled auto-numbered\DUfootnotemark{footnote-reference-3}{label}{2}, or symbolic\DUfootnotemark{footnote-reference-4}{footnote-3}{*}), citation references (see \hyperlink{cit2002}{[CIT2002]}),
substitution references (\includegraphics{../../../docs/user/rst/images/biohazard.png} \&
a \emph{trimmed heart} \texttt{(U+2665):}♥), and %
\phantomsection\label{inline-hyperlink-targets}inline hyperlink targets
(see \hyperref[targets]{Targets} below for a reference back to here).  Character-level
inline markup is also possible (although exceedingly ugly!) in \emph{re}\texttt{Structured}\emph{Text}.  Problems are indicated by %
\raisebox{1em}{\hypertarget{problematic-1}{}}\hyperlink{system-message-1}{\textbf{\color{red}|problematic|}} text
(generated by processing errors; this one is intentional).  Here is a
reference to the \hyperref[doctitle]{doctitle} and the \hyperref[subtitle]{subtitle}.

The default role for interpreted text is \DUroletitlereference{Title Reference}.  Here are
some explicit interpreted text roles: a PEP reference (\href{https://peps.python.org/pep-0287}{PEP 287}); an
RFC reference (\href{https://tools.ietf.org/html/rfc2822.html}{RFC 2822}); an abbreviation (\DUrole{abbreviation}{abb.}), an acronym
(\DUrole{acronym}{reST}), code (\texttt{\DUrole{code}{print \textquotedbl{}hello world\textquotedbl{}}}); a \textsubscript{subscript};
a \textsuperscript{superscript} and explicit roles for \DUroletitlereference{Docutils}’
\emph{standard} \textbf{inline} \texttt{markup}.

% DO NOT RE-WRAP THE FOLLOWING PARAGRAPH!

Let’s test wrapping and whitespace significance in inline literals:
\texttt{This is an example of --inline-literal --text, --including some--
strangely--hyphenated-words. ~Adjust-the-width-of-your-browser-window
to see how the text is wrapped. ~-- ---- -------- ~Now note ~ ~the
spacing ~ ~between the ~ ~words of ~ ~this sentence ~ ~(words
should ~ ~be grouped ~ ~in pairs).}

If the \texttt{--pep-references} option was supplied, there should be a
live link to PEP 258 here.


\subsection{2.2   Bullet Lists%
  \label{bullet-lists}%
}

\begin{itemize}
\item A bullet list

\begin{itemize}
\item Nested bullet list.

\item Nested item 2.
\end{itemize}

\item Item 2.

Paragraph 2 of item 2.

\begin{itemize}
\item Nested bullet list.

\item Nested item 2.

\begin{itemize}
\item Third level.

\item Item 2.
\end{itemize}

\item Nested item 3.

\item This nested list should be compacted by the HTML writer.

\phantomsection\label{target}
% Even if this item contains a target and a comment.
\end{itemize}
\end{itemize}


\subsection{2.3   Enumerated Lists%
  \label{enumerated-lists}%
}

\begin{enumerate}
\item Arabic numerals.

\begin{enumerate}
\renewcommand{\labelenumii}{\alph{enumii})}
\item lower alpha)

\begin{enumerate}
\renewcommand{\labelenumiii}{(\roman{enumiii})}
\item (lower roman)

\begin{enumerate}
\item upper alpha.

\begin{list}{\Roman{enumv})}{\usecounter{enumv}}
\item upper roman)
\end{list}
\end{enumerate}
\end{enumerate}
\end{enumerate}

\item Lists that don’t start at 1:

\begin{enumerate}
\renewcommand{\labelenumii}{\arabic{enumii}.}
\setcounter{enumii}{2}
\item Three

\item Four
\end{enumerate}

\begin{enumerate}
\renewcommand{\labelenumii}{\Alph{enumii}.}
\setcounter{enumii}{2}
\item C

\item D
\end{enumerate}

\begin{enumerate}
\renewcommand{\labelenumii}{\roman{enumii}.}
\setcounter{enumii}{2}
\item iii

\item iv
\end{enumerate}
\end{enumerate}


\subsection{2.4   Definition Lists%
  \label{definition-lists}%
}

\begin{description}
\item[{Term}] 
Definition

\item[{Term}] (\textbf{classifier})
Definition paragraph 1.

Definition paragraph 2.

\item[{Term}] 
Definition

\item[{Term}] (\textbf{classifier one})(\textbf{classifier two})
Definition
\end{description}


\subsection{2.5   Field Lists%
  \label{field-lists}%
}

\begin{DUfieldlist}
\item[{what:}]
Field lists map field names to field bodies, like database
records.  They are often part of an extension syntax.  They are
an unambiguous variant of RFC 2822 fields.

\item[{how arg1 arg2:}]
The field marker is a colon, the field name, and a colon.

The field body may contain one or more body elements, indented
relative to the field marker.

\item[{credits:}]
\DUrole{credits}{This paragraph has the \DUroletitlereference{credits} class set.  (This is actually not
about credits but just for ensuring that the class attribute
doesn’t get stripped away.)}
\end{DUfieldlist}


\subsection{2.6   Option Lists%
  \label{option-lists}%
}

For listing command-line options:

\begin{DUoptionlist}
\item[-a]  command-line option “a”
\item[-b file]  options can have arguments
and long descriptions
\item[--long]  options can be long, too
\item[--input=file]  long options can also have arguments
\item[/V]  DOS/VMS-style option
\item[--very-long-option]  The description can also start on the next line.

The description may contain multiple body elements,
regardless of where it starts.
\item[-x, -y, -z]  Multiple options are an “option group”.
\item[-v, --verbose]  Commonly-seen: short \& long options.
\item[-1 file, --one=file, --two file]  Multiple options with arguments.
\item[-f <{[}path{]}file>]  Option argumens must start with a letter
or be wrapped in angle brackets.
\item[-d <src dest>]  Angle brackets are also required if an
option expects more than one argument.
\end{DUoptionlist}

There must be at least two spaces between the option and the
description.


\subsection{2.7   Literal Blocks%
  \label{literal-blocks}%
}

Literal blocks are indicated with a double-colon (“::”) at the end of
the preceding paragraph (over there \texttt{-->}).  They can be indented:

\begin{quote}
\begin{alltt}
if literal_block:
    text = 'is left as-is'
    spaces_and_linebreaks = 'are preserved'
    markup_processing = None
\end{alltt}
\end{quote}

Or they can be quoted without indentation:

\begin{quote}
\begin{alltt}
>> Great idea!
>
> Why didn't I think of that?
\end{alltt}
\end{quote}


\subsection{2.8   Line Blocks%
  \label{line-blocks}%
}

This section tests line blocks.  Line blocks are body elements which
consist of lines and other line blocks.  Nested line blocks cause
indentation.

\begin{DUlineblock}{0em}
\item[] This is a line block.  It ends with a blank line.
\item[]
\begin{DUlineblock}{\DUlineblockindent}
\item[] New lines begin with a vertical bar (“|”).
\item[] Line breaks and initial indent are significant, and preserved.
\item[]
\begin{DUlineblock}{\DUlineblockindent}
\item[] Continuation lines are also possible.  A long line that is intended
to wrap should begin with a space in place of the vertical bar.
\end{DUlineblock}
\item[] The left edge of a continuation line need not be aligned with
the left edge of the text above it.
\end{DUlineblock}
\end{DUlineblock}

\begin{DUlineblock}{0em}
\item[] This is a second line block.
\item[] 
\item[] Blank lines are permitted internally, but they must begin with a “|”.
\end{DUlineblock}

Another line block, surrounded by paragraphs:

\begin{DUlineblock}{0em}
\item[] And it’s no good waiting by the window
\item[] It’s no good waiting for the sun
\item[] Please believe me, the things you dream of
\item[] They don’t fall in the lap of no-one
\end{DUlineblock}

Take it away, Eric the Orchestra Leader!

\begin{quote}
\begin{DUlineblock}{0em}
\item[] A one, two, a one two three four
\item[] 
\item[] Half a bee, philosophically,
\item[]
\begin{DUlineblock}{\DUlineblockindent}
\item[] must, \emph{ipso facto}, half not be.
\end{DUlineblock}
\item[] But half the bee has got to be,
\item[]
\begin{DUlineblock}{\DUlineblockindent}
\item[] \emph{vis a vis} its entity.  D’you see?
\item[] 
\end{DUlineblock}
\item[] But can a bee be said to be
\item[]
\begin{DUlineblock}{\DUlineblockindent}
\item[] or not to be an entire bee,
\item[]
\begin{DUlineblock}{\DUlineblockindent}
\item[] when half the bee is not a bee,
\item[]
\begin{DUlineblock}{\DUlineblockindent}
\item[] due to some ancient injury?
\item[] 
\end{DUlineblock}
\end{DUlineblock}
\end{DUlineblock}
\item[] Singing…
\end{DUlineblock}
\end{quote}

A line block, like the following poem by Christian Morgenstern, can
also be centre-aligned:

\begin{selectlanguage}{german}
\begin{DUlineblock}{0em}
\centering
\item[] \textbf{Die Trichter}
\item[] 
\item[] Zwei Trichter wandeln durch die Nacht.
\item[] Durch ihres Rumpfs verengten Schacht
\item[] fließt weißes Mondlicht
\item[] still und heiter
\item[] auf   ihren
\item[] Waldweg
\item[] u. s.
\item[] w.
\item[] 
\end{DUlineblock}
\end{selectlanguage}


\subsection{2.9   Block Quotes%
  \label{block-quotes}%
}

Block quotes consist of indented body elements:

\begin{quote}
My theory by A. Elk.  Brackets Miss, brackets.  This theory goes
as follows and begins now.  All brontosauruses are thin at one
end, much much thicker in the middle and then thin again at the
far end.  That is my theory, it is mine, and belongs to me and I
own it, and what it is too.
\nopagebreak

\raggedleft —Anne Elk (Miss)
\end{quote}

The language of a quote (like any other object) can be specified by
a class attribute:

% 

\begin{selectlanguage}{french}
\begin{quote}
ReStructuredText est un langage de balisage léger utilisé
notamment dans la documentation du langage Python.
\end{quote}
\end{selectlanguage}


\subsection{2.10   Doctest Blocks%
  \label{doctest-blocks}%
}

\begin{quote}
\begin{alltt}
>>> print 'Python-specific usage examples; begun with ">>>"'
Python-specific usage examples; begun with ">>>"
>>> print '(cut and pasted from interactive Python sessions)'
(cut and pasted from interactive Python sessions)
\end{alltt}
\end{quote}


\subsection{2.11   Footnotes%
  \label{footnotes}%
}
%
\DUfootnotetext{footnote-1}{footnote-reference-1}{1}{%
A footnote contains body elements, consistently indented by at
least 1 space.

This is the footnote’s second paragraph.
}
%
\DUfootnotetext{label}{footnote-reference-3}{2}{\phantomsection\label{label}%
Footnotes may be numbered, either manually (as in\DUfootnotemark{footnote-reference-5}{footnote-1}{1}) or
automatically using a “\#”-prefixed label.  This footnote has a
label so it can be referred to from multiple places, both as a
footnote reference (\DUfootnotemark{footnote-reference-6}{label}{2}) and as a \hyperref[label]{hyperlink reference}.
}
%
\DUfootnotetext{footnote-2}{footnote-reference-2}{3}{%
This footnote is numbered automatically and anonymously using a
label of “\#” only.

This is the second paragraph.

And this is the third paragraph.
}
%
\DUfootnotetext{footnote-3}{footnote-reference-4}{*}{%
Footnotes may also use symbols, specified with a “*” label.
Here’s a reference to the next footnote:\DUfootnotemark{footnote-reference-7}{footnote-4}{†}.
}
%
\DUfootnotetext{footnote-4}{footnote-reference-7}{†}{%
This footnote shows the next symbol in the sequence.
}
%
\DUfootnotetext{footnote-5}{footnote-5}{4}{%
Here’s an unreferenced footnote, with a reference to a
nonexistent footnote:%
\raisebox{1em}{\hypertarget{footnote-reference-8}{}}\hyperlink{system-message-2}{\textbf{\color{red}{[}5{]}\_}}.
}


\subsection{2.12   Citations%
  \label{citations}%
}
\begin{figure}[b]\raisebox{1em}{\hypertarget{cit2002}{}}[CIT2002]
Citations are text-labeled footnotes. They may be
rendered separately and differently from footnotes.
\end{figure}

Here’s a reference to the above, \hyperlink{cit2002}{[CIT2002]}, and a %
\raisebox{1em}{\hypertarget{citation-reference-3}{}}\hyperlink{system-message-3}{\textbf{\color{red}{[}nonexistent{]}\_}}
citation.


\subsection{2.13   Targets%
  \label{targets}%
  \label{another-target}%
}

\phantomsection\label{example}
This paragraph is pointed to by the explicit “example” target. A
reference can be found under \hyperref[inline-markup]{Inline Markup}, above. \hyperref[inline-hyperlink-targets]{Inline
hyperlink targets} are also possible.

Section headers are implicit targets, referred to by name. See
\hyperref[targets]{Targets}, which is a subsection of \hyperref[body-elements]{Body Elements}.

Explicit external targets are interpolated into references such as
“\href{http://www.python.org/}{Python}\DUfootnotemark{footnote-reference-11}{footnote-6}{5}”.

Targets may be indirect and anonymous.  Thus \hyperref[targets]{this phrase} may also
refer to the \hyperref[targets]{Targets} section.

Here’s a %
\raisebox{1em}{\hypertarget{problematic-2}{}}\hyperlink{system-message-4}{\textbf{\color{red}`hyperlink reference without a target`\_}}, which generates an
error.


\subsubsection{2.13.1   Duplicate Target Names%
  \label{duplicate-target-names}%
}

Duplicate names in section headers or other implicit targets will
generate “info” (level-1) system messages.  Duplicate names in
explicit targets will generate “warning” (level-2) system messages.


\subsubsection{2.13.2   Duplicate Target Names%
  \label{duplicate-target-names-1}%
}

Since there are two “Duplicate Target Names” section headers, we
cannot uniquely refer to either of them by name.  If we try to (like
this: %
\raisebox{1em}{\hypertarget{problematic-3}{}}\hyperlink{system-message-5}{\textbf{\color{red}`Duplicate Target Names`\_}}), an error is generated.


\subsection{2.14   Directives%
  \label{directives}%
}

\phantomsection\label{contents}

\begin{DUclass}{contents}
\begin{DUclass}{local}

\begin{DUclass}{auto-toc}
\begin{itemize}
\item \hyperref[document-parts]{2.14.1   Document Parts}

\item \hyperref[images-and-figures]{2.14.2   Images and Figures}

\item \hyperref[tables]{2.14.3   Tables}

\item \hyperref[admonitions]{2.14.4   Admonitions}

\item \hyperref[topics-sidebars-and-rubrics]{2.14.5   Topics, Sidebars, and Rubrics}

\item \hyperref[target-footnotes]{2.14.6   Target Footnotes}

\item \hyperref[replacement-text]{2.14.7   Replacement Text}

\item \hyperref[compound-paragraph]{2.14.8   Compound Paragraph}

\item \hyperref[parsed-literal-blocks]{2.14.9   Parsed Literal Blocks}

\item \hyperref[code]{2.14.10   Code}

\item \hyperref[meta]{2.14.11   Meta}
\end{itemize}
\end{DUclass}
\end{DUclass}
\end{DUclass}

These are just a sample of the many reStructuredText Directives.  For
others, please see \href{https://docutils.sourceforge.io/docs/ref/rst/directives.html}{reStructuredText Directives}\DUfootnotemark{footnote-reference-19}{footnote-11}{10}.


\subsubsection{2.14.1   Document Parts%
  \label{document-parts}%
}

An example of the “contents” directive can be seen above this section
(a local, untitled table of \hyperref[contents]{contents}) and at the beginning of the
document (a document-wide \hyperref[table-of-contents]{table of contents}).


\subsubsection{2.14.2   Images and Figures%
  \label{images-and-figures}%
}

An image directive (also clickable – a hyperlink reference):

\hyperref[directives]{\includegraphics[width=0.7\linewidth]{../../../docs/user/rst/images/title.png}}

Image with multiple IDs:

\includegraphics{../../../docs/user/rst/images/biohazard.png}
\phantomsection\label{image-target-3}\label{image-target-2}\label{image-target-1}

A centered image:

\noindent\makebox[\linewidth][c]{\includegraphics{../../../docs/user/rst/images/biohazard.png}}

A left-aligned image:

\noindent{\includegraphics{../../../docs/user/rst/images/biohazard.png}\hfill}

This paragraph might flow around the image.
The specific behavior depends upon the style sheet and
the browser or rendering software used.

A right-aligned image:

\noindent{\hfill\includegraphics{../../../docs/user/rst/images/biohazard.png}}

This paragraph might flow around the image.
The specific behavior depends upon the style sheet and
the browser or rendering software used.

For inline images see \hyperref[substitution-definitions]{Substitution Definitions}.

Image size:

An image 2 em wide:

\includegraphics[width=2em]{../../../docs/user/rst/images/biohazard.png}

An image 2 cm wide and 15 pixel high:

\includegraphics[height=15\pdfpxdimen,width=2cm]{../../../docs/user/rst/images/biohazard.png}

Relative units allow adaption of the image to the screen or paper size.
An image occupying 50\% of the line width:

\includegraphics[width=0.5\linewidth]{../../../docs/user/rst/images/title.png}

A \emph{figure} is an image with a caption and/or a legend.  With page-based output
media, figures might float to a different position if this helps the page
layout.

\begin{DUclass}{figclass1}
\begin{DUclass}{figclass2}
\begin{figure}
\noindent\makebox[\linewidth][c]{\href{https://docutils.sourceforge.io/rst.html}{\includegraphics[width=258bp]{../../../docs/user/rst/images/title.png}}}
\caption{Plaintext markup syntax and parser system.}
\begin{DUlegend}
\setlength{\DUtablewidth}{\linewidth}%
\begin{longtable*}{|p{0.156\DUtablewidth}|p{0.563\DUtablewidth}|}
\hline

re
 & 
Revised, revisited, based on ‘re’ module.
 \\
\hline

Structured
 & 
Structure-enhanced text, structuredtext.
 \\
\hline

Text
 & 
Well it is, isn’t it?
 \\
\hline
\end{longtable*}

This paragraph is also part of the legend.
\end{DUlegend}
\end{figure}
\end{DUclass}
\end{DUclass}

A left-aligned figure, 70\% wide:

\begin{DUclass}{figclass1}
\begin{DUclass}{figclass2}
\begin{figure} % align = "left"
\noindent\makebox[\linewidth][c]{\includegraphics[width=40\pdfpxdimen]{../../../docs/user/rst/images/biohazard.png}}
\caption{\label{caption-label}\DUrole{captionclass1}{\DUrole{captionclass2}{This is the caption.}}}
\begin{DUlegend}
This is the legend.

The legend may consist of several paragraphs.
\end{DUlegend}
\end{figure}
\end{DUclass}
\end{DUclass}

This paragraph might flow around the figure.

The specific behavior depends upon the style sheet and the browser or
rendering software used.

A centred figure, 40\% wide:

\begin{figure}
\noindent\makebox[\linewidth][c]{\includegraphics[width=40\pdfpxdimen]{../../../docs/user/rst/images/biohazard.png}}
\caption{This is the caption.}
\begin{DUlegend}
This is the legend.

The legend may consist of several paragraphs.
\end{DUlegend}
\end{figure}

This paragraph might flow around the figure.

The specific behavior depends upon the style sheet and the browser or
rendering software used.

A right-aligned figure:

\begin{figure} % align = "right"
\noindent\makebox[\linewidth][c]{\includegraphics[width=40\pdfpxdimen]{../../../docs/user/rst/images/biohazard.png}}
\caption{This is the caption.}
\begin{DUlegend}
This is the legend.

The legend may consist of several paragraphs.
\end{DUlegend}
\end{figure}

This paragraph might flow around the figure. The specific behavior depends
upon the style sheet and the browser or rendering software used.


\subsubsection{2.14.3   Tables%
  \label{tables}%
}

Tables may be given titles and additional arguments with the \emph{table}
directive:

\setlength{\DUtablewidth}{\linewidth}%
\begin{longtable}[l]{|p{0.075\DUtablewidth}|p{0.075\DUtablewidth}|}
\caption{left-aligned table}\\
\hline
\textbf{%
A
} & \textbf{%
not A
} \\
\hline
\endfirsthead
\caption[]{left-aligned table (... continued)}\\
\hline
\textbf{%
A
} & \textbf{%
not A
} \\
\hline
\endhead
\multicolumn{2}{p{0.15\DUtablewidth}}{\raggedleft\ldots continued on next page}\\
\endfoot
\endlastfoot

False
 & 
True
 \\
\hline

True
 & 
False
 \\
\hline
\end{longtable}

\setlength{\DUtablewidth}{\linewidth}%
\begin{longtable}[c]{|p{0.075\DUtablewidth}|p{0.075\DUtablewidth}|}
\caption{center-aligned table}\\
\hline
\textbf{%
A
} & \textbf{%
not A
} \\
\hline
\endfirsthead
\caption[]{center-aligned table (... continued)}\\
\hline
\textbf{%
A
} & \textbf{%
not A
} \\
\hline
\endhead
\multicolumn{2}{p{0.15\DUtablewidth}}{\raggedleft\ldots continued on next page}\\
\endfoot
\endlastfoot

False
 & 
True
 \\
\hline

True
 & 
False
 \\
\hline
\end{longtable}

\setlength{\DUtablewidth}{\linewidth}%
\begin{longtable}[r]{|p{0.075\DUtablewidth}|p{0.075\DUtablewidth}|}
\caption{right-aligned table}\\
\hline
\textbf{%
A
} & \textbf{%
not A
} \\
\hline
\endfirsthead
\caption[]{right-aligned table (... continued)}\\
\hline
\textbf{%
A
} & \textbf{%
not A
} \\
\hline
\endhead
\multicolumn{2}{p{0.15\DUtablewidth}}{\raggedleft\ldots continued on next page}\\
\endfoot
\endlastfoot

False
 & 
True
 \\
\hline

True
 & 
False
 \\
\hline
\end{longtable}

With the “widths” argument “auto” (or “class” value “colwidths-auto”),
column widths are determined by the backend (if supported by the
writer/backend).

\phantomsection\label{target2}\label{target1}
\begin{longtable*}{|l|l|l|}
\hline
\textbf{A} & \textbf{B} & \textbf{A or B} \\
\hline
\endfirsthead
\hline
\textbf{A} & \textbf{B} & \textbf{A or B} \\
\hline
\endhead
\endfoot
\endlastfoot
False & False & False \\
\hline
True & False & True \\
\hline
False & True & True \\
\hline
True & True & True \\
\hline
\end{longtable*}


\subsubsection{2.14.4   Admonitions%
  \label{admonitions}%
}

\begin{DUclass}{attention}
\begin{DUadmonition}
\DUtitle{Attention!}

Directives at large.
\end{DUadmonition}
\end{DUclass}

\begin{DUclass}{caution}
\begin{DUadmonition}
\DUtitle{Caution!}

Don’t take any wooden nickels.
\end{DUadmonition}
\end{DUclass}

\begin{DUclass}{danger}
\begin{DUadmonition}
\DUtitle{!DANGER!}

Mad scientist at work!
\end{DUadmonition}
\end{DUclass}

\begin{DUclass}{error}
\begin{DUadmonition}
\DUtitle{Error}

Does not compute.
\end{DUadmonition}
\end{DUclass}

\begin{DUclass}{hint}
\begin{DUadmonition}
\DUtitle{Hint}

It’s bigger than a bread box.
\end{DUadmonition}
\end{DUclass}

\begin{DUclass}{important}
\begin{DUadmonition}
\DUtitle{Important}

\begin{itemize}
\item Wash behind your ears.

\item Clean up your room.

\item Call your mother.

\item Back up your data.
\end{itemize}
\end{DUadmonition}
\end{DUclass}

\begin{DUclass}{note}
\begin{DUadmonition}
\DUtitle{Note}

This is a note.
\end{DUadmonition}
\end{DUclass}

\begin{DUclass}{tip}
\begin{DUadmonition}
\DUtitle{Tip}

15\% if the service is good.
\end{DUadmonition}
\end{DUclass}

\begin{DUclass}{warning}
\begin{DUadmonition}
\DUtitle{Warning}

Strong prose may provoke extreme mental exertion.
Reader discretion is strongly advised.
\end{DUadmonition}
\end{DUclass}

\begin{DUclass}{admonition-and-by-the-way}
\begin{DUadmonition}
\DUtitle{And, by the way…}

You can make up your own admonition too.
\end{DUadmonition}
\end{DUclass}


\subsubsection{2.14.5   Topics, Sidebars, and Rubrics%
  \label{topics-sidebars-and-rubrics}%
}

\emph{Sidebars} are like miniature, parallel documents.

\DUsidebar{
\DUtitle{Optional Sidebar Title}

\DUsubtitle{Optional Subtitle}

This is a sidebar.  It is for text outside the flow of the main
text.

\DUrubric{This is a rubric inside a sidebar}

Sidebars often appear beside the main text with a border and a different
background or font color.
}

A \emph{topic} is like a block quote with a title, or a self-contained section
with no subsections.

\begin{DUclass}{topic}
\begin{quote}
\DUtitle{Topic Title}

This is a topic.
\end{quote}
\end{DUclass}

A \emph{rubric} is like an informal heading that doesn’t correspond to the
document’s structure. It is typically highlighted in red (hence the name).

\DUrubric{This is a rubric}

Topics and rubrics can be used at places where a \hyperref[section-title]{section title} is not
allowed (e.g. inside a directive).


\subsubsection{2.14.6   Target Footnotes%
  \label{target-footnotes}%
}
%
\DUfootnotetext{footnote-6}{footnote-reference-10}{5}{%
\url{http://www.python.org/}
}
%
\DUfootnotetext{footnote-7}{footnote-reference-13}{6}{%
\url{http://pygments.org/}
}
%
\DUfootnotetext{footnote-8}{footnote-reference-14}{7}{%
\url{ftp://ftp.ams.org/ams/doc/amsmath/short-math-guide.pdf}
}
%
\DUfootnotetext{footnote-9}{footnote-reference-16}{8}{%
\url{http://ctan.org/pkg/polyglossia}
}
%
\DUfootnotetext{footnote-10}{footnote-reference-18}{9}{%
\url{https://docutils.sourceforge.io/}
}
%
\DUfootnotetext{footnote-11}{footnote-reference-19}{10}{%
\url{https://docutils.sourceforge.io/docs/ref/rst/directives.html}
}
%
\DUfootnotetext{footnote-12}{footnote-reference-20}{11}{%
\url{https://docutils.sourceforge.io/docs/ref/rst/directives.html\#metadata}
}


\subsubsection{2.14.7   Replacement Text%
  \label{replacement-text}%
}

I recommend you try \href{http://www.python.org/}{Python, \emph{the} best language around}\DUfootnotemark{footnote-reference-12}{footnote-6}{5}.


\subsubsection{2.14.8   Compound Paragraph%
  \label{compound-paragraph}%
}

The \emph{compound} directive is used to create a “compound paragraph”, which
is a single logical paragraph containing multiple physical body
elements. For example:

\begin{DUclass}{compound}
The ‘rm’ command is very dangerous.  If you are logged
in as root and enter
\begin{quote}
\begin{alltt}
cd /
rm -rf *
\end{alltt}
\end{quote}
you will erase the entire contents of your file system.
\end{DUclass}

Test the handling and display of compound paragraphs:

\begin{DUclass}{compound}
\begin{DUclass}{some-class}
Compound 2, paragraph 1,

compound 2, paragraph 2,
\begin{itemize}
\item list item 1,

\item list item 2,
\end{itemize}
compound 2, paragraph 3.
\end{DUclass}
\end{DUclass}

\begin{DUclass}{compound}
Compound 3, only consisting of one paragraph.
\end{DUclass}

\begin{DUclass}{compound}
\begin{quote}
\begin{alltt}
Compound 4.
This one starts with a literal block.
\end{alltt}
\end{quote}
Compound 4, paragraph following the literal block.
\end{DUclass}

Now something \emph{really} perverted – a nested compound block.  This is
just to test that it works at all; the results don’t have to be
meaningful.

\begin{DUclass}{compound}
Compound 5, block 1 (a paragraph).

\begin{DUclass}{compound}
Compound 6 is block 2 in compound 5.

Compound 6, another paragraph.
\end{DUclass}

Compound 5, block 3 (a paragraph).
\end{DUclass}

\begin{DUclass}{compound}
Compound 7, tests the inclusion of various block-level
elements in one logical paragraph. First a table,
\setlength{\DUtablewidth}{\linewidth}%
\begin{longtable*}{|p{0.249\DUtablewidth}|p{0.249\DUtablewidth}|p{0.249\DUtablewidth}|}
\hline

Left cell, first
paragraph.

Left cell, second
paragraph.
 & 
Middle cell,
consisting of
exactly one
paragraph.
 & 
Right cell.

Paragraph 2.

Paragraph 3.
 \\
\hline
\end{longtable*}
followed by a paragraph. This physical paragraph is
actually a continuation of the paragraph before the table. It is followed
by
\begin{quote}
a quote and
\end{quote}
\begin{enumerate}
\item an enumerated list,
\end{enumerate}
a paragraph,
\begin{DUoptionlist}
\item[--an]  option list,
\end{DUoptionlist}
a paragraph,
\begin{DUfieldlist}
\item[{a field:}]
list,
\end{DUfieldlist}
a paragraph,
\begin{description}
\item[{a definition}] 
list,
\end{description}
a paragraph, an image:
\includegraphics{../../../docs/user/rst/images/biohazard.png}
a paragraph,
\begin{DUlineblock}{0em}
\item[] a line
\item[] block,
\end{DUlineblock}
a paragraph followed by a comment,
% this is a comment
a paragraph, a
\begin{DUclass}{note}
\begin{DUadmonition}
\DUtitle{Note}

with content
\end{DUadmonition}
\end{DUclass}
and the final paragraph of the compound 7.
\end{DUclass}


\subsubsection{2.14.9   Parsed Literal Blocks%
  \label{parsed-literal-blocks}%
}

\begin{quote}
\ttfamily\raggedright
This~is~a~parsed~literal~block.\\
~~~~This~line~is~indented.~~The~next~line~is~blank.\\
~\\
Inline~markup~is~supported,~e.g.~\emph{emphasis},~\textbf{strong},~\texttt{literal\\
text},~\textsubscript{sub-}~and~\textsuperscript{super}scripts,\\
inline~formulas:~$A = 2 \pi r^2$,\\
footnotes\DUfootnotemark{footnote-reference-9}{footnote-1}{1},~%
\phantomsection\label{hyperlink-targets}hyperlink~targets,~and~\href{http://www.python.org/}{references}.
\end{quote}


\subsubsection{2.14.10   Code%
  \label{code}%
}

Blocks of source code can be set with the \DUroletitlereference{code} directive. If the code
language is specified, the content is parsed and tagged by the \href{http://pygments.org/}{Pygments}\DUfootnotemark{footnote-reference-13}{footnote-7}{6}
syntax highlighter and can be formatted with a style sheet. (Code parsing
is turned off using the \texttt{syntax-highlight} config setting in the test
conversions in order to get identical results with/without installed
Pygments highlighter.)

\begin{DUclass}{code}
\begin{DUclass}{python}
\begin{quote}
\begin{alltt}
print 'This is Python code.'
\end{alltt}
\end{quote}
\end{DUclass}
\end{DUclass}

The \texttt{:number-lines:} option (with optional start value) generates line
numbers:

\begin{DUclass}{code}
\begin{DUclass}{python}
\begin{quote}
\ttfamily\raggedright
\DUrole{ln}{~8~}\#~print~integers~from~0~to~9:\\
\DUrole{ln}{~9~}for~i~in~range(10):\\
\DUrole{ln}{10~}~~~~print~i
\end{quote}
\end{DUclass}
\end{DUclass}

For inline code snippets, there is the \DUroletitlereference{code} role, which can be used
directly (the code will not be parsed/tagged, as the language is not known)
or as base for special code roles, e.g. the LaTeX code in the next
paragraph.

Docutils uses LaTeX syntax for math directives and roles:
\texttt{\DUrole{code}{\DUrole{tex}{\textbackslash{}alpha = f(x)}}} prints $\alpha = f(x)$.

The \texttt{:code:} option of the \DUroletitlereference{include} directive sets the included content
as a code block, here the rst file \texttt{header\_footer.rst} with line numbers:

\begin{DUclass}{code}
\begin{DUclass}{rst}
\begin{quote}
\ttfamily\raggedright
\DUrole{ln}{1~}..~header::~Document~header\\
\DUrole{ln}{2~}..~footer::~Document~footer
\end{quote}
\end{DUclass}
\end{DUclass}


\subsubsection{2.14.11   Meta%
  \label{meta}%
}

The \href{https://docutils.sourceforge.io/docs/ref/rst/directives.html\#metadata}{“meta” directive}\DUfootnotemark{footnote-reference-20}{footnote-12}{11} is used to specify metadata to be stored in,
e.g., HTML META tags or ODT file properties.


\subsection{2.15   Substitution Definitions%
  \label{substitution-definitions}%
}

An inline image (\includegraphics{../../../docs/user/rst/images/biohazard.png}) example:

A Unicode example:

(Substitution definitions are only visible in the rST source.)


\subsection{2.16   Comments%
  \label{comments}%
}

Here’s one:

% Comments begin with two dots and a space. Anything may
% follow, except for the syntax of footnotes, hyperlink
% targets, directives, or substitution definitions.
% 
% Double-dashes -- "--" -- must be escaped somehow in HTML output.
% 
% Comments may contain non-ASCII characters: ä ö ü æ ø å

(View the HTML/LaTeX/… source to see the comment.)


\subsection{2.17   Raw text%
  \label{raw-text}%
}

This does not necessarily look nice, because there may be missing white space.

It’s just there to freeze the behavior.

A test.

Second test.

\DUrole{myclass}{Another test with myclass set.}

This is the \DUrole{myrawroleclass}{fourth test} with myrawroleclass set.

Fifth test in LaTeX.\\Line two.


\subsection{2.18   Container%
  \label{container}%
}

\begin{DUclass}{custom}
paragraph 1

paragraph 2
\end{DUclass}

% currently not implemented in LaTeX:
% .. include:: data/header_footer.rst


\subsection{2.19   Colspanning tables%
  \label{colspanning-tables}%
}

This table has a cell spanning two columns:

\setlength{\DUtablewidth}{\linewidth}%
\begin{longtable*}{|p{0.075\DUtablewidth}|p{0.075\DUtablewidth}|p{0.086\DUtablewidth}|}
\hline
\multicolumn{2}{|p{0.15\DUtablewidth}|}{\textbf{%
Inputs
}} & \textbf{%
Output
} \\
\hline
\textbf{%
A
} & \textbf{%
B
} & \textbf{%
A or B
} \\
\hline
\endfirsthead
\hline
\multicolumn{2}{|p{0.15\DUtablewidth}|}{\textbf{%
Inputs
}} & \textbf{%
Output
} \\
\hline
\textbf{%
A
} & \textbf{%
B
} & \textbf{%
A or B
} \\
\hline
\endhead
\multicolumn{3}{p{0.24\DUtablewidth}}{\raggedleft\ldots continued on next page}\\
\endfoot
\endlastfoot

False
 & 
False
 & 
False
 \\
\hline

True
 & 
False
 & 
True
 \\
\hline

False
 & 
True
 & 
True
 \\
\hline

True
 & 
True
 & 
True
 \\
\hline
\end{longtable*}


\subsection{2.20   Rowspanning tables%
  \label{rowspanning-tables}%
}

Here’s a table with cells spanning several rows:

\setlength{\DUtablewidth}{\linewidth}%
\begin{longtable*}{|p{0.296\DUtablewidth}|p{0.156\DUtablewidth}|p{0.226\DUtablewidth}|}
\hline
\textbf{%
Header row, column 1
(header rows optional)
} & \textbf{%
Header 2
} & \textbf{%
Header 3
} \\
\hline
\endfirsthead
\hline
\textbf{%
Header row, column 1
(header rows optional)
} & \textbf{%
Header 2
} & \textbf{%
Header 3
} \\
\hline
\endhead
\multicolumn{3}{p{0.68\DUtablewidth}}{\raggedleft\ldots continued on next page}\\
\endfoot
\endlastfoot

body row 1, column 1
 & 
column 2
 & 
column 3
 \\
\hline

body row 2
 & \multirow{2}{0.16\DUtablewidth}{%
Cells may
span rows.
} & \multirow{2}{0.23\DUtablewidth}{%
Another
rowspanning
cell.
} \\
\cline{1-1}

body row 3
 &  &  \\
\hline
\end{longtable*}


\subsection{2.21   List Tables%
  \label{list-tables}%
}

Here’s a list table exercising all features:

\begin{DUclass}{test}
\setlength{\DUtablewidth}{0.95\linewidth}%
\begin{longtable}{|p{0.133\DUtablewidth}|p{0.110\DUtablewidth}|p{0.249\DUtablewidth}|}
\caption{list table with integral header}\\
\hline
\textbf{%
Treat
} & \textbf{%
Quantity
} & \textbf{%
Description
} \\
\hline
\endfirsthead
\caption[]{list table with integral header (... continued)}\\
\hline
\textbf{%
Treat
} & \textbf{%
Quantity
} & \textbf{%
Description
} \\
\hline
\endhead
\multicolumn{3}{p{0.49\DUtablewidth}}{\raggedleft\ldots continued on next page}\\
\endfoot
\endlastfoot
\textbf{%
Albatross
} & 
2.99
 & 
On a stick!
 \\
\hline
\textbf{%
Crunchy Frog
} & 
1.49
 & 
If we took the bones out, it wouldn’t be
crunchy, now would it?
 \\
\hline
\textbf{%
Gannet Ripple
} & 
1.99
 & 
On a stick!
 \\
\hline
\end{longtable}
\end{DUclass}

\begin{longtable}[c]{|l|l|}
\caption{center aligned list table}\\
\hline
Albatross & 2.99 \\
\hline
Crunchy Frog & 1.49 \\
\hline
Gannet Ripple & 1.99 \\
\hline
\end{longtable}


\subsection{2.22   Custom Roles%
  \label{custom-roles}%
}

\begin{itemize}
\item A role based on an existing role.

\texttt{\DUrole{custom}{one}} \texttt{\DUrole{custom}{two}} \texttt{\DUrole{custom}{three}}

\item A new role.

\DUrole{customnew}{one two three}

\item A role with class attribute.

\DUrole{special}{interpreted text}

\item A language-switching role:

Let’s count in German \foreignlanguage{german}{eins zwei drei}.

\item A role with multiple class attributes, styled with raw directives:

\newcommand{\DUrolegreen}[1]{\textcolor{green}{#1}}
\newcommand{\DUrolesc}[1]{\textsc{#1}}

The following works in most browsers but does not validate
(\texttt{<style>} is only allowed in the document head):

\begin{quote}
\begin{alltt}
.. raw:: html

  <style type="text/css"><!--
   .green \{color: green;\}
   .sc \{font-variant: small-caps;\}
   --></style>
\end{alltt}
\end{quote}

\DUrole{green}{\DUrole{sc}{\foreignlanguage{english}{British colourful text in small-caps}}}.
\end{itemize}


\subsection{2.23   Mathematics%
  \label{mathematics}%
}

Docutils supports inline math with the prefix or postfix \texttt{:math:}
role specificator, $n! + \sin(x_n^2)$ and $A_\text{c} =
\frac{\pi}{4} d^2$, as well as displayed math via the
\DUroletitlereference{math} directive:
%
\begin{equation*}
f(\epsilon) = \frac{1}{1 + \exp\left(\frac{\varepsilon}{k_\text{B}T}\right)}
\end{equation*}
Content may start on the first line of the directive, e.g.
%
\begin{equation*}
N = \frac{\text{number of apples}}{7}
\end{equation*}
Equations can be labeled with a reference name using the \texttt{:name:} option.
See \hyperref[eq-m]{eq:M} and \hyperref[eq-schrodinger]{eq:schrödinger} below.

The determinant of the matrix
%
\begin{equation*}
\mathbf{M} = \left(\begin{matrix}a&b\\c&d\end{matrix}\right)
\label{eq-m}
\end{equation*}
is $|\mathbf{M}| = ad - bc$.

More than one display math block can be put in one math directive.
For example, the following sum and integral with limits:
%
\begin{equation*}
\int_0^1 x^n dx = \frac{1}{n + 1}
\end{equation*}%
\begin{equation*}
\sum_{n=1}^m n = \frac{m(m+1)}{2}
\end{equation*}
LaTeX-supported Unicode math symbols can be used in math roles and
directives:

The Schrödinger equation
%
\begin{equation*}
i\hbar \frac{\partial }{\partial t}\Psi  = \hat{H}\Psi ,
\label{eq-schrodinger}
\end{equation*}
with the \emph{wave function} $\Psi $, describes how the quantum state of a
physical system changes in time.

\begin{description}
\item[{Math-Accents:}] \leavevmode
\setlength{\DUtablewidth}{1\linewidth}%
\begin{longtable*}{p{0.315\DUtablewidth}p{0.315\DUtablewidth}p{0.315\DUtablewidth}}

$\acute{a}$      \texttt{\textbackslash{}acute\{a\}}
 & 
$\dot{t}$        \texttt{\textbackslash{}dot\{t\}}
 & 
$\hat{\gamma}$   \texttt{\textbackslash{}hat\{\textbackslash{}gamma\}}
 \\

$\grave{a}$      \texttt{\textbackslash{}grave\{a\}}
 & 
$\ddot{t}$       \texttt{\textbackslash{}ddot\{t\}}
 & 
$\tilde{\alpha}$ \texttt{\textbackslash{}tilde\{\textbackslash{}alpha\}}
 \\

$\breve{x}$      \texttt{\textbackslash{}breve\{x\}}
 & 
$\dddot{t}$      \texttt{\textbackslash{}dddot\{t\}}
 & 
$\vec{\imath}$   \texttt{\textbackslash{}vec\{\textbackslash{}imath\}}
 \\

$\check{a}$      \texttt{\textbackslash{}check\{a\}}
 & 
$\bar{a}$        \texttt{\textbackslash{}bar\{a\}}
 & 
$\vec{R}$        \texttt{\textbackslash{}vec\{R\}}
 \\
\end{longtable*}
\end{description}

% \widetilde{xxx}
% \widehat{xxx}

Modulation Transfer Function:
%
\begin{equation*}
\text{MTF} = \left|\frac{\mathcal{F}\{s(x)\}}
                        {\mathcal{F}\{ s(x)\} |_{\omega _{x}=0}}\right|
           = \mathrm{abs}\left(\frac
                 {\int _{-\infty }^{\infty }s(x) \mathrm{e}^{\mathrm{i}\omega _{x}x}\mathrm{d}{x}}
                 {\int _{-\infty }^{\infty }s(x)\mathrm{d}{x}}
             \right).
\end{equation*}
Math split over two lines: If a double backslash is detected outside a
\texttt{\textbackslash{}begin\{...\} \textbackslash{}end\{...\}} pair, the math code is wrapped in an \href{ftp://ftp.ams.org/ams/doc/amsmath/short-math-guide.pdf}{AMSmath}\DUfootnotemark{footnote-reference-14}{footnote-8}{7}
\texttt{align} environment:
%
\begin{align*}
s_{\mathrm{out}}(x) & = s_{\mathrm{in}}(x') * s_\delta (x-x') \\
                    & = \int  s_{\mathrm{in}}(x')s_\delta (x-x')\mathrm{d}x'
\end{align*}
Cases with the \href{ftp://ftp.ams.org/ams/doc/amsmath/short-math-guide.pdf}{AMSmath}\DUfootnotemark{footnote-reference-15}{footnote-8}{7} \texttt{cases} environment:
%
\begin{equation*}
\mathrm{sgn}(x) = \begin{cases}
                             -1 & x<0\\
                    \phantom{-} 1 & x>0
                  \end{cases}
\end{equation*}

\section{3   Tests for the LaTeX writer%
  \label{tests-for-the-latex-writer}%
}

Test syntax elements which may cause trouble for the LaTeX writer but might
not need to be tested with other writers. See also the
“latex\_cornercases” test.


\subsection{3.1   Custom Roles in LaTeX%
  \label{custom-roles-in-latex}%
}

\begin{itemize}
\item Role names and class arguments are converted to conform to the
regular expression \texttt{{[}a-z{]}{[}-a-z0-9{]}*} (letters are downcased,
accents and similar decoration is stripped, non-conforming
characters are replaced by a hyphen).

Class arguments may contain numbers and hyphens, which need special
treatment in LaTeX command names.

\DUrole{large}{\DUrole{custom4}{\DUrole{small-caps}{\DUrole{custom-role}{\DUrole{custom-role}{Text with role “custom4”}}}}} (but without styling by \texttt{DUrole*}
macros).

\item With LaTeX, roles can be styled within the document using the \DUroletitlereference{raw}
directive.

\newcommand{\DUrolelarge}[1]{{\large #1}}
\makeatletter
\@namedef{DUrolesmall-caps}{\textsc}
\@namedef{DUrolecustom4}{\textbf}
\makeatother

\DUrole{large}{\DUrole{custom4}{\DUrole{small-caps}{\DUrole{custom-role}{\DUrole{custom-role}{inline text}}}}} in large, bold, small-caps.

\item Custom roles can be based on standard roles:

This is a \emph{\DUrole{custom-emphasis}{customized emphasis text role}}

This is a \texttt{\DUrole{custom-literal}{customized literal text role}}

This is a \textbf{\DUrole{custom-strong}{customized strong text role}}

This is a \textsubscript{\DUrole{custom-subscript}{customized subscript text role}}

This is a \textsuperscript{\DUrole{custom-superscript}{customized superscript text role}}

This is a \DUroletitlereference{\DUrole{custom-title-reference}{customized title-reference text role}}
\end{itemize}


\subsection{3.2   class handling%
  \label{class-handling}%
}

This section tests class handling for block level elements by the LaTeX
writer. See the input file \texttt{classes\_latex.rst} for the raw LaTeX code used
to style the examples.

An “epigraph” directive is exported as “quote” wrapped in a “DUclass”
environment. Here, it is styled by a “DUCLASSepigraph” environment
redefining the “quote” environment as “minipage”:

\newcommand*{\DUCLASSepigraph}{%
  \renewenvironment{quote}{\vspace{1em}
                           \footnotesize\hfill{}%
                           \begin{minipage}{0.4\columnwidth}}%
                          {\end{minipage}\vskip\baselineskip}}

\begin{DUclass}{epigraph}
\begin{quote}
Do not play this piece fast. It is never right to play \emph{Ragtime} fast.
\nopagebreak

\raggedleft —Scott Joplin
\end{quote}
\end{DUclass}

Raw latex is also used to style the following lists: “DUCLASSenumerateitems”
redefines “itemize” as “enumerate”, “DUCLASSrules” draws horizontal lines
above and below.

\newcommand*{\DUCLASSenumerateitems}{%
  \renewenvironment{itemize}{\begin{enumerate}}%
                            {\end{enumerate}}%
}

\newenvironment{DUCLASSrules}%
               {\noindent\rule[0.5ex]{1\columnwidth}{1pt}}%
               {\noindent\rule[0.5ex]{1\columnwidth}{1pt}}

An “enumerated” bullet list:

\begin{DUclass}{enumerateitems}
\begin{itemize}
\item item

\item next item

\item third item
\end{itemize}
\end{DUclass}

A list with lines above and below:

\begin{DUclass}{rules}
\begin{itemize}
\item item

\item next item
\end{itemize}
\end{DUclass}

A normal bullet list is kept unchanged by the above redefinitions:

\begin{itemize}
\item item

\item next item

\item third item
\end{itemize}

A container wraps several elements in a common “class wrapper”. Here, we use
it to set 2 paragraphs and a list in small caps:

\newcommand*{\DUCLASSscshape}{\scshape}

\begin{DUclass}{scshape}
paragraph 1

paragraph 2

\begin{itemize}
\item bullet list

\item still bullet list
\end{itemize}
\end{DUclass}

A right-aligned line-block. Alignment handling is built into the latex
writer for image, table, and line block elements.

\begin{DUlineblock}{0em}
\raggedleft
\item[] Max Mustermann
\item[] Waldstr. 22
\item[] D 01234 Testdorf
\item[] Tel.: 0123/456789
\end{DUlineblock}

% 


\section{4   Tests for the XeTeX writer%
  \label{tests-for-the-xetex-writer}%
}

With XeTeX, you can typeset text in any language/script supported by
Unicode and the selected font, e.g. Azərbaycanca, Bân-lâm-gú, Башҡорт
Беларуская, Български, Català, Чӑвашла, Česky, Ελληνικά, Español,
Français, Føroyskt, Хальмг, Íslenska, עברית , Қазақша, Kurdî,
Latviešu, Lietuvių, Македонски, Монгол, Nāhuatl, Português, Română,
Русский, Slovenščina, Српски, Türkçe, Українська, Tiếng Việt, Volapük,
Võro, ייִדיש , Žemaitėška.
There is extended support for more than 40 languages in the
\href{http://ctan.org/pkg/polyglossia}{polyglossia}\DUfootnotemark{footnote-reference-16}{footnote-9}{8} package.

% preface for System Messages:


\section{5   Error Handling%
  \label{error-handling}%
}

Any errors caught during processing will generate system messages.

There should be five messages in the following, auto-generated
section, “Docutils System Messages”:

% section should be added by Docutils automatically


\section[Docutils System Messages]{\color{red}Docutils System Messages%
}

\raisebox{1em}{\hypertarget{system-message-1}{}}
\begin{DUclass}{system-message}
\begin{DUadmonition}
\DUtitle{\color{red}System Message}

ERROR/3 in \texttt{functional/input/data/standard.rst}, line~99

\hyperlink{problematic-1}{
Undefined substitution referenced: \textquotedbl{}problematic\textquotedbl{}.
}
\end{DUadmonition}
\end{DUclass}

\raisebox{1em}{\hypertarget{system-message-2}{}}
\begin{DUclass}{system-message}
\begin{DUadmonition}
\DUtitle{\color{red}System Message}

ERROR/3 in \texttt{functional/input/data/standard.rst}, line~390

\hyperlink{footnote-reference-8}{
Unknown target name: \textquotedbl{}5\textquotedbl{}.
}
\end{DUadmonition}
\end{DUclass}

\raisebox{1em}{\hypertarget{system-message-3}{}}
\begin{DUclass}{system-message}
\begin{DUadmonition}
\DUtitle{\color{red}System Message}

ERROR/3 in \texttt{functional/input/data/standard.rst}, line~399

\hyperlink{citation-reference-3}{
Unknown target name: \textquotedbl{}nonexistent\textquotedbl{}.
}
\end{DUadmonition}
\end{DUclass}

\raisebox{1em}{\hypertarget{system-message-4}{}}
\begin{DUclass}{system-message}
\begin{DUadmonition}
\DUtitle{\color{red}System Message}

ERROR/3 in \texttt{functional/input/data/standard.rst}, line~426

\hyperlink{problematic-2}{
Unknown target name: \textquotedbl{}hyperlink reference without a target\textquotedbl{}.
}
\end{DUadmonition}
\end{DUclass}

\raisebox{1em}{\hypertarget{system-message-5}{}}
\begin{DUclass}{system-message}
\begin{DUadmonition}
\DUtitle{\color{red}System Message}

ERROR/3 in \texttt{functional/input/data/standard.rst}, line~439

\hyperlink{problematic-3}{
Duplicate target name, cannot be used as a unique reference: \textquotedbl{}duplicate target names\textquotedbl{}.
}
\end{DUadmonition}
\end{DUclass}

\end{document}
