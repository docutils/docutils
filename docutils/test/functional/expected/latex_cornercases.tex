\documentclass[a4paper]{article}
% generated by Docutils <https://docutils.sourceforge.io/>
\usepackage{cmap} % fix search and cut-and-paste in Acrobat
\usepackage[T1]{fontenc}
\DeclareUnicodeCharacter{21D4}{\ensuremath{\Leftrightarrow}}
\DeclareUnicodeCharacter{2660}{\ensuremath{\spadesuit}}
\DeclareUnicodeCharacter{2663}{\ensuremath{\clubsuit}}
\usepackage{alltt}
\usepackage{amsmath}
\usepackage{float} % extended float configuration
\floatplacement{figure}{H} % place figures here definitely
\usepackage{graphicx}
\usepackage{multirow}
\usepackage{pifont}
\setcounter{secnumdepth}{0}
\usepackage{longtable,ltcaption,array}
\setlength{\extrarowheight}{2pt}
\newlength{\DUtablewidth} % internal use in tables
\newcommand{\DUcolumnwidth}[1]{\dimexpr#1\DUtablewidth-2\tabcolsep\relax}
\usepackage{textcomp} % text symbol macros

%%% Custom LaTeX preamble
% PDF Standard Fonts
\usepackage{mathptmx} % Times
\usepackage[scaled=.90]{helvet}
\usepackage{courier}

%%% User specified packages and stylesheets

%%% Fallback definitions for Docutils-specific commands

% class handling for environments (block-level elements)
% \begin{DUclass}{spam} tries \DUCLASSspam and
% \end{DUclass}{spam} tries \endDUCLASSspam
\ifdefined\DUclass
\else % poor man's "provideenvironment"
  \newenvironment{DUclass}[1]%
    {% "#1" does not work in end-part of environment.
     \def\DocutilsClassFunctionName{DUCLASS#1}
     \csname \DocutilsClassFunctionName \endcsname}%
    {\csname end\DocutilsClassFunctionName \endcsname}%
\fi

% Provide a length variable and set default, if it is new
\providecommand*{\DUprovidelength}[2]{%
  \ifdefined#1
  \else
    \newlength{#1}\setlength{#1}{#2}%
  \fi
}

% admonition environment (specially marked topic)
\ifdefined\DUadmonition
\else % poor man's "provideenvironment"
  \newbox{\DUadmonitionbox}
  \newenvironment{DUadmonition}%
    {\begin{center}
       \begin{lrbox}{\DUadmonitionbox}
         \begin{minipage}{0.9\linewidth}
    }%
    {    \end{minipage}
       \end{lrbox}
       \fbox{\usebox{\DUadmonitionbox}}
     \end{center}
    }
\fi

% legend environment (in figures and formal tables)
\ifdefined\DUlegend
\else
  \newenvironment{DUlegend}{\small}{}
\fi

% line block environment
\DUprovidelength{\DUlineblockindent}{2.5em}
\ifdefined\DUlineblock
\else
  \newenvironment{DUlineblock}[1]{%
    \list{}{\setlength{\partopsep}{\parskip}
            \addtolength{\partopsep}{\baselineskip}
            \setlength{\topsep}{0pt}
            \setlength{\itemsep}{0.15\baselineskip}
            \setlength{\parsep}{0pt}
            \setlength{\leftmargin}{#1}}
    \raggedright
  }
  {\endlist}
\fi

% list of command line options
\providecommand*{\DUoptionlistlabel}[1]{\bfseries #1 \hfill}
\DUprovidelength{\DUoptionlistindent}{3cm}
\ifdefined\DUoptionlist
\else
  \newenvironment{DUoptionlist}{%
    \list{}{\setlength{\labelwidth}{\DUoptionlistindent}
            \setlength{\rightmargin}{1cm}
            \setlength{\leftmargin}{\rightmargin}
            \addtolength{\leftmargin}{\labelwidth}
            \addtolength{\leftmargin}{\labelsep}
            \renewcommand{\makelabel}{\DUoptionlistlabel}}
  }
  {\endlist}
\fi

% title for topics, admonitions, unsupported section levels, and sidebar
\providecommand*{\DUtitle}[1]{%
  \smallskip\noindent\textbf{#1}\smallskip}

% titlereference standard role
\providecommand*{\DUroletitlereference}[1]{\textsl{#1}}

% transition (break / fancybreak / anonymous section)
\providecommand*{\DUtransition}{%
  \hspace*{\fill}\hrulefill\hspace*{\fill}
  \vskip 0.5\baselineskip
}

% hyperlinks:
\ifdefined\hypersetup
\else
  \usepackage[hyperfootnotes=false,
              colorlinks=true,linkcolor=blue,urlcolor=blue]{hyperref}
  \usepackage{bookmark}
  \urlstyle{same} % normal text font (alternatives: tt, rm, sf)
\fi
\hypersetup{
  pdftitle={Additional Tests for the LaTeX Writer},
}

%%% Body
\begin{document}
\title{Additional Tests for the LaTeX Writer%
  \label{additional-tests-for-the-latex-writer}}
\author{}
\date{}
\maketitle

These tests contain syntax elements and combinations which may cause
trouble for the LaTeX writer.

\pdfbookmark[1]{Contents}{contents}
\tableofcontents


\section{Section heading levels%
  \label{section-heading-levels}%
}


\section{Level 1%
  \label{level-1}%
}

Nested sections


\subsection{Level 2%
  \label{level-2}%
}

reach at some level


\subsubsection{Level 3%
  \label{level-3}%
}

(depending on the document class and output format)


\paragraph{level 4%
  \label{level-4}%
}

a level


\subparagraph{level 5%
  \label{level-5}%
}

that is not supported by the output format.


\begin{DUclass}{sectionVI}
\DUtitle{level 6%
  \label{level-6}%
}
\end{DUclass}

Unsupported in LaTeX and HTML5
(HTML5 reserves the 1st level for the document title).


\begin{DUclass}{sectionVII}
\DUtitle{level 7%
  \label{level-7}%
}
\end{DUclass}

Unsupported in HTML4.


\begin{DUclass}{sectionVIII}
\DUtitle{level 8%
  \label{level-8}%
}
\end{DUclass}

Unsupported in ODT.


\section{Section titles with inline markup%
  \label{section-titles-with-inline-markup}%
  \label{references}%
}


\subsection{\emph{emphasized}, H\textsubscript{2}O, $x^2$, and \hyperref[references]{references}%
  \label{emphasized-h2o-x-2-and-references}%
}


\subsection{Substitutions work%
  \label{substitutions-fail}%
}

Note, that the \textquotedbl{}reference name\textquotedbl{} for this section is derived from the
content \emph{before} substitution. You can link to it with the \href{https://docutils.sourceforge.io/docs/ref/rst/restructuredtext.html\#hyperlink-references}{phrase
reference} \textquotedbl{}\hyperref[substitutions-fail]{substitutions fail}\textquotedbl{}.
This behaviour may be exploited to get intelligible IDs after \href{https://docutils.sourceforge.io/docs/ref/rst/directives.html\#identifier-normalization}{identifier
normalization} of the section's reference name.

% This file is used by the standalone_rst_latex test.


\section{Option lists%
  \label{option-lists}%
}

The LaTeX-2e description environment is used for definition lists.
The definition is continued on the same line as the term, this should
not happen if an option-list is at the top of the definition.

If the option list is not at the first element in the definition, it
is contained in a quote

\begin{quote}
\begin{DUoptionlist}
\item[-{}-help]  show help
\item[-v]  verbose
\end{DUoptionlist}
\end{quote}

\begin{description}
\item[{In a definition list:}] \leavevmode
\begin{DUoptionlist}
\item[-{}-help]  show help
\item[-v]  verbose
\end{DUoptionlist}
\end{description}


\section{Block Quotes%
  \label{block-quotes}%
}

\begin{quote}
This block quote comes directly after the section heading and is
followed by a paragraph.

This is the second paragraph of the block quote and it contains
some more text filling up some space which would otherwise be
empty.
\nopagebreak

\raggedleft —Attribution
\end{quote}

This is a paragraph.

\begin{quote}
This block quote does not have an attribution.
\end{quote}

This is another paragraph.

\begin{quote}
Another block quote at the end of the section.
\end{quote}


\section{More Block Quotes%
  \label{more-block-quotes}%
}

\begin{quote}
Block quote followed by a transition.
\end{quote}

%___________________________________________________________________________
\DUtransition

\begin{quote}
Another block quote.
\end{quote}


\section{Images%
  \label{images}%
}

Image with 20\% width:

\includegraphics[width=0.2\linewidth]{../../../docs/user/rst/images/title.png}

Image with 100\% width:

\includegraphics[width=1\linewidth]{../../../docs/user/rst/images/title.png}


\section{Tables%
  \label{tables}%
}

In contrast to HTML, LaTeX does not support line-breaks in tables with
\textquotedbl{}automatic\textquotedbl{} column widths. Each cell has just one line, paragraphs are
merged (the writer emits a warning).

\begin{longtable}{|l|l|}
\caption{problems with \textquotedbl{}auto\textquotedbl{} widths}\\
\hline
11 & first paragraph
second paragraph \\
\hline
content
with
linebreak & 22 \\
\hline
\end{longtable}

To provide for arbitrary cell content, the LaTeX writer defaults to
specifying column widths computed from the source column widths. This
works sufficiently in many cases:

\setlength{\DUtablewidth}{\dimexpr\linewidth-7\arrayrulewidth\relax}%
\begin{longtable}{|p{\DUcolumnwidth{0.153}}|p{\DUcolumnwidth{0.208}}|p{\DUcolumnwidth{0.208}}|p{\DUcolumnwidth{0.125}}|p{\DUcolumnwidth{0.139}}|p{\DUcolumnwidth{0.167}}|}
\caption{a table with multi-paragraph multi-column cells}\\
\hline

test
 & 
\textbf{bold hd}
 & \multicolumn{3}{p{\DUcolumnwidth{0.472}}|}{%
multicolumn 1

With a second paragraph
} & 
\emph{emph hd}
 \\
\hline
\multicolumn{2}{|p{\DUcolumnwidth{0.361}}|}{%
multicolumn 2

With a second paragraph
} & 
cell
 & 
cell
 & 
cell
 & 
cell
 \\
\hline

cell
 & \multicolumn{2}{p{\DUcolumnwidth{0.417}}|}{%
multicolumn 3 (one line,
but very very very very
very looooong)
} & 
cell
 & 
cell
 & 
cell
 \\
\hline

cell
 & 
cell
 & 
cell
 & \multicolumn{3}{p{\DUcolumnwidth{0.431}}|}{%
Short multicolumn 4
} \\
\hline
\end{longtable}

A problem with the source-derived column widths is that simple tables
often use no padding while grid tables without padding look cramped:

\setlength{\DUtablewidth}{\dimexpr\linewidth-5\arrayrulewidth\relax}%
\begin{longtable}{|p{\DUcolumnwidth{0.100}}|p{\DUcolumnwidth{0.050}}|p{\DUcolumnwidth{0.050}}|p{\DUcolumnwidth{0.050}}|}
\caption{simple table, not padded in the source}\\
\hline
\textbf{%
A
} & \textbf{%
B
} & \textbf{%
C
} & \textbf{%
D
} \\
\hline
\endfirsthead
\caption[]{simple table, not padded in the source (... continued)}\\
\hline
\textbf{%
A
} & \textbf{%
B
} & \textbf{%
C
} & \textbf{%
D
} \\
\hline
\endhead
\multicolumn{4}{p{\DUcolumnwidth{0.250}}}{\raggedleft\ldots continued on next page}\\
\endfoot
\endlastfoot

100
 & 
2
 & 
3
 & 
4
 \\
\hline

EUR
 & 
b
 & 
c
 & 
d
 \\
\hline
\end{longtable}

\setlength{\DUtablewidth}{\dimexpr\linewidth-5\arrayrulewidth\relax}%
\begin{longtable}{|p{\DUcolumnwidth{0.150}}|p{\DUcolumnwidth{0.100}}|p{\DUcolumnwidth{0.100}}|p{\DUcolumnwidth{0.100}}|}
\caption{grid table, padded cells}\\
\hline
\textbf{%
A
} & \textbf{%
B
} & \textbf{%
C
} & \textbf{%
D
} \\
\hline
\endfirsthead
\caption[]{grid table, padded cells (... continued)}\\
\hline
\textbf{%
A
} & \textbf{%
B
} & \textbf{%
C
} & \textbf{%
D
} \\
\hline
\endhead
\multicolumn{4}{p{\DUcolumnwidth{0.450}}}{\raggedleft\ldots continued on next page}\\
\endfoot
\endlastfoot

100
 & 
2
 & 
3
 & 
4
 \\
\hline

EUR
 & 
b
 & 
c
 & 
d
 \\
\hline
\end{longtable}

For better typographic results, setting the \DUroletitlereference{width} and/or
\DUroletitlereference{widths} options of the \href{https://docutils.sourceforge.io/docs/ref/rst/directives.html\#table}{table directive} is recommended.

\begin{longtable}{|l|l|l|l|}
\caption{grid table, auto-width columns}\\
\hline
\textbf{A} & \textbf{B} & \textbf{C} & \textbf{D} \\
\hline
\endfirsthead
\caption[]{grid table, auto-width columns (... continued)}\\
\hline
\textbf{A} & \textbf{B} & \textbf{C} & \textbf{D} \\
\hline
\endhead
\endfoot
\endlastfoot
100 & 2 & 3 & 4 \\
\hline
EUR & b & c & d \\
\hline
\end{longtable}

\begin{longtable}{|l|l|}
\caption{table with multi-row header and \textquotedbl{}auto\textquotedbl{} column-widths}\\
\hline
\multirow{2}{*}{\textbf{XXX}} & \textbf{Variable Summary} \\
\cline{2-2}
 & \textbf{Description} \\
\hline
\endfirsthead
\caption[]{table with multi-row header and \textquotedbl{}auto\textquotedbl{} column-widths (... continued)}\\
\hline
\multirow{2}{*}{\textbf{XXX}} & \textbf{Variable Summary} \\
\cline{2-2}
 & \textbf{Description} \\
\hline
\endhead
\multicolumn{2}{l}{\raggedleft\ldots continued on next page}\\
\endfoot
\endlastfoot
\multicolumn{2}{|l|}{multi-column cell} \\
\hline
\end{longtable}

The \DUroletitlereference{width} option overrides \textquotedbl{}auto\textquotedbl{} \DUroletitlereference{widths} as standard LaTeX tables
don't have a global width setting:

\setlength{\DUtablewidth}{\dimexpr0.6\linewidth-5\arrayrulewidth\relax}%
\begin{longtable}{|p{\DUcolumnwidth{0.400}}|p{\DUcolumnwidth{0.200}}|p{\DUcolumnwidth{0.200}}|p{\DUcolumnwidth{0.200}}|}
\caption{This table has \DUroletitlereference{widths} \textquotedbl{}auto\textquotedbl{} (ignored) and \DUroletitlereference{width} 60\%.}\\
\hline
\textbf{%
A
} & \textbf{%
B
} & \textbf{%
C
} & \textbf{%
D
} \\
\hline
\endfirsthead
\caption[]{This table has \DUroletitlereference{widths} \textquotedbl{}auto\textquotedbl{} (ignored) and \DUroletitlereference{width} 60\%. (... continued)}\\
\hline
\textbf{%
A
} & \textbf{%
B
} & \textbf{%
C
} & \textbf{%
D
} \\
\hline
\endhead
\multicolumn{4}{p{\DUcolumnwidth{1.000}}}{\raggedleft\ldots continued on next page}\\
\endfoot
\endlastfoot

100
 & 
2
 & 
3
 & 
4
 \\
\hline

EUR
 & 
b
 & 
c
 & 
d
 \\
\hline
\end{longtable}


\subsection{Nested tables%
  \label{nested-tables}%
}

\setlength{\DUtablewidth}{\dimexpr\linewidth-3\arrayrulewidth\relax}%
\begin{longtable*}{|p{\DUcolumnwidth{0.700}}|p{\DUcolumnwidth{0.300}}|}
\hline

Lorem ipsum dolor sit amet, consectetur
 & 
adipisicing elit
 \\
\hline

\label{nested-table}
\noindent\makebox[\linewidth][r]{%
\setlength{\DUtablewidth}{\dimexpr\linewidth-3\arrayrulewidth\relax}%
\begin{tabular}{|p{\DUcolumnwidth{0.150}}|p{\DUcolumnwidth{0.150}}|}
\hline

1
 & 
2
 \\
\hline
\end{tabular}
}
 & 
cell 1, 2
 \\
\hline

table width depends on parent column

\noindent\makebox[\linewidth][c]{%
\setlength{\DUtablewidth}{\dimexpr\linewidth-3\arrayrulewidth\relax}%
\begin{tabular}{|p{\DUcolumnwidth{0.150}}|p{\DUcolumnwidth{0.150}}|}
\hline

1
 & 
2
 \\
\hline
\end{tabular}
}

better use \textquotedbl{}auto\textquotedbl{} widths, see below
 & 
same table

\setlength{\DUtablewidth}{\dimexpr\linewidth-3\arrayrulewidth\relax}%
\begin{tabular}{|p{\DUcolumnwidth{0.150}}|p{\DUcolumnwidth{0.150}}|}
\hline

1
 & 
2
 \\
\hline
\end{tabular}

in narrow column
 \\
\hline

\noindent\makebox[\linewidth][r]{%
\begin{tabular}{|l|l|}
\hline
1 & 2 \\
\hline
\end{tabular}
}

\begin{description}
\item[{definition:}] 
list
\end{description}
 & 
cell 3, 2
 \\
\hline
\end{longtable*}


\subsection{TODO%
  \label{todo}%
}

\begin{itemize}
\item Tables with multi-paragraph multi-row cells currently fail due to a
LaTeX limitation (see \url{https://sourceforge.net/p/docutils/bugs/225/}).

\item Tweak vertical spacing in table cells containing multiple elements.
\end{itemize}

See also \texttt{test/functional/input/data/latex-problematic.rst}.


\section{Monospaced non-alphanumeric characters%
  \label{monospaced-non-alphanumeric-characters}%
}

These are all ASCII characters except a-zA-Z0-9 and space:

\texttt{!!!\textquotedbl{}\textquotedbl{}\textquotedbl{}\#\#\#\$\$\$\%\%\%\&\&\&'{}'{}'((()))***+++,{},{},-{}-{}-...///:::}

\texttt{;;;<{}<{}<===>{}>{}>???@@@{[}{[}{[}\textbackslash{}\textbackslash{}\textbackslash{}{]}{]}{]}\textasciicircum{}\textasciicircum{}\textasciicircum{}\_\_\_`{}`{}`\{\{\{|||\}\}\}\textasciitilde{}\textasciitilde{}\textasciitilde{}}

\texttt{xxxxxxxxxxxxxxxxxxxxxxxxxxxxxxxxxxxxxxxxxxxxxxxx}

The two lines of non-alphanumeric characters should both have the same
width as the third line.


\section{Non-ASCII characters%
  \label{non-ascii-characters}%
}

\begin{longtable}{|l|l|}
\caption{Punctuation and footnote symbols}\\
\hline
– & en-dash \\
\hline
— & em-dash \\
\hline
‘ & single turned comma quotation mark \\
\hline
’ & single comma quotation mark \\
\hline
‚ & low single comma quotation mark \\
\hline
“ & double turned comma quotation mark \\
\hline
” & double comma quotation mark \\
\hline
„ & low double comma quotation mark \\
\hline
† & dagger \\
\hline
‡ & double dagger \\
\hline
\ding{169} & black diamond suit \\
\hline
\ding{170} & black heart suit \\
\hline
♠ & black spade suit \\
\hline
♣ & black club suit \\
\hline
… & ellipsis \\
\hline
™ & trade mark sign \\
\hline
⇔ & left-right double arrow \\
\hline
\end{longtable}

\begin{longtable}{|l|l|l|l|l|l|l|l|l|l|l|}
\caption{The \DUroletitlereference{Latin-1 extended} Unicode block}\\
\hline

% 
 & 0 & 1 & 2 & 3 & 4 & 5 & 6 & 7 & 8 & 9 \\
\hline
160 &  & ¡ & ¢ & £ &  & ¥ & ¦ & § & ¨ & © \\
\hline
170 & ª & « & ¬ & \- & ® & ¯ & ° & ± & ² & ³ \\
\hline
180 & ´ & µ & ¶ & · & ¸ & ¹ & º & » & ¼ & ½ \\
\hline
190 & ¾ & ¿ & À & Á & Â & Ã & Ä & Å & Æ & Ç \\
\hline
200 & È & É & Ê & Ë & Ì & Í & Î & Ï & Ð & Ñ \\
\hline
210 & Ò & Ó & Ô & Õ & Ö & × & Ø & Ù & Ú & Û \\
\hline
220 & Ü & Ý & Þ & ß & à & á & â & ã & ä & å \\
\hline
230 & æ & ç & è & é & ê & ë & ì & í & î & ï \\
\hline
240 & ð & ñ & ò & ó & ô & õ & ö & ÷ & ø & ù \\
\hline
250 & ú & û & ü & ý & þ & ÿ &  &  &  &  \\
\hline
\end{longtable}

\begin{itemize}
\item The following line should not be wrapped, because it uses
no-break spaces (\textbackslash{}u00a0):

X X X X X X X X X X X X X X X X X X X X X X X X X X X X X X X X X X X X X X X X X X X X X X X X X X X X X X X X X X X X X X X X X X

\item Line wrapping with/without breakpoints marked by soft hyphens
(\textbackslash{}u00ad):

pdn\-derd\-mdtd\-ri\-schpdn\-derd\-mdtd\-ri\-schpdn\-derd\-mdtd\-ri\-schpdn\-derd\-mdtd\-ri\-schpdn\-derd\-mdtd\-ri\-sch

pdnderdmdtdrischpdnderdmdtdrischpdnderdmdtdrischpdnderdmdtdrischpdnderdmdtdrisch
\end{itemize}


\section{Encoding special characters%
  \label{encoding-special-characters}%
}

The LaTeX Info pages list under \textquotedbl{}2.18 Special Characters\textquotedbl{}

\begin{quote}
The following characters play a special role in LaTeX and are called
\textquotedbl{}special printing characters\textquotedbl{}, or simply \textquotedbl{}special characters\textquotedbl{}.

\begin{quote}
\# \$ \% \& \textasciitilde{} \_ \textasciicircum{} \textbackslash{} \{ \}
\end{quote}
\end{quote}

The special chars verbatim:

\begin{quote}
\begin{alltt}
# $ % & ~ _ ^ \textbackslash{} \{ \}
\end{alltt}
\end{quote}

However also \emph{square brackets} {[}{]} need special care.

\begin{quote}
Commands with optional arguments (e.g. \texttt{\textbackslash{}item}) check
if the token right after the macro name is an opening bracket.
In that case the contents between that bracket and the following
closing bracket on the same grouping level are taken as the
optional argument. What makes this unintuitive is the fact that
the square brackets aren't grouping characters themselves, so in
your last example \texttt{\textbackslash{}item{[}{[}...{]}{]}} the optional argument consists of
{[}... (without the closing bracket).
\end{quote}

Compare the items in the following lists:

\begin{itemize}
\item simple item

\item {[}bracketed{]} item
\end{itemize}

\begin{description}
\item[{simple}] 
description term

\item[{{[}bracketed{]}}] 
description term
\end{description}

The OT1 font-encoding differs from ASCII for the less-than, greater-than
and bar characters (< | >) except for typewriter font \DUroletitlereference{cmtt}
(\texttt{< | >}).


\section{Hyperlinks and -targets%
  \label{hyperlinks-and-targets}%
}

In LaTeX, we must set an explicit anchor (\texttt{\textbackslash{}phantomsection}) for a
%
\phantomsection\label{hypertarget-in-plain-text}hypertarget in plain text or in a figure but not in a table title
or figure caption:

\setlength{\DUtablewidth}{\dimexpr\linewidth-4\arrayrulewidth\relax}%
\begin{longtable}{|p{\DUcolumnwidth{0.150}}|p{\DUcolumnwidth{0.150}}|p{\DUcolumnwidth{0.150}}|}
\caption{Table with %
\label{hypertarget-in-table-title}hypertarget in table title.\label{table-label}}\\
\hline

False
 & 
True
 & 
None
 \\
\hline
\end{longtable}

\phantomsection\label{figure-label}
\begin{figure}
\noindent\makebox[\linewidth][c]{\includegraphics{../../../docs/user/rst/images/biohazard.png}}
\caption{Figure with %
\label{hypertarget-in-figure-caption}hypertarget in figure caption.}
\begin{DUlegend}
Legend with %
\phantomsection\label{hypertarget-in-figure-legend}hypertarget in figure legend.
\end{DUlegend}
\end{figure}

\phantomsection\label{image-label}
\includegraphics{../../../docs/user/rst/images/biohazard.png}

See \hyperref[hypertarget-in-plain-text]{hypertarget in plain text},
\hyperref[table-label]{table label}, \hyperref[hypertarget-in-table-title]{hypertarget in table title},
\hyperref[nested-table]{nested table},
\hyperref[figure-label]{figure label}, \hyperref[hypertarget-in-figure-caption]{hypertarget in figure caption},
\hyperref[hypertarget-in-figure-legend]{hypertarget in figure legend}, and
\hyperref[image-label]{image label}.


\section{External references%
  \label{external-references}%
}

Long URLs should be wrapped in the PDF. This can be achieved with the
\texttt{\textbackslash{}url} command which is used by the LaTeX writer whenever the content
(name) of a reference node equals the link URL.

\begin{description}
\item[{Example:}] 
a long URL that should wrap in the output
\url{https://docutils.sourceforge.io/docs/user/latex.html\#id79}
\end{description}

If the argument contains any \textquotedbl{}\%\textquotedbl{}, \textquotedbl{}\#\textquotedbl{}, or \textquotedbl{}\textasciicircum{}\textasciicircum{}\textquotedbl{}, or ends with \texttt{\textbackslash{}}, it can't
be used in the argument to another command. The argument must not contain
unbalanced braces.

The characters \textasciicircum{}, \{, \}, and \texttt{\textbackslash{}} are invalid in a \textquotedbl{}http:\textquotedbl{} or \textquotedbl{}ftp:\textquotedbl{} URL
and not recognized as part of it:

\begin{DUlineblock}{0em}
\item[] \url{http://www.example.org}/strange\textasciicircum{}\textasciicircum{}name
\item[] \url{http://www.example.org}\textbackslash{}using\textbackslash{}DOS\textbackslash{}paths\textbackslash{}
\item[] \url{http://www.example.org/XML}/strange\{n\}ame
\end{DUlineblock}

They can, however be used in paths and/or filenames.

Handling by the LaTeX writer:

\begin{itemize}
\item \texttt{\#}, \texttt{\textbackslash{}} and \texttt{\%} are escaped:

\begin{DUlineblock}{0em}
\item[] \href{http://www.w3.org/XML/Schema\#dev}{URL with \#}
\url{http://www.w3.org/XML/Schema\#dev}
\item[] \href{http://www.w3.org/XML/Schema\%dev}{URL with \%}
\url{http://example.org/Schema\%dev}
\item[] \href{A:DOS\\path\\}{file with DOS path} \url{A:DOS\\path\\}
\end{DUlineblock}

\begin{DUclass}{note}
\begin{DUadmonition}
\DUtitle{Note}

These URLs are typeset inside a LaTeX command without error.

\begin{DUlineblock}{0em}
\item[] \url{http://www.w3.org/XML/Schema\#dev}
\item[] \url{http://example.org/Schema\%dev}
\item[] \url{A:DOS\\path\\}
\end{DUlineblock}
\end{DUadmonition}
\end{DUclass}
\end{itemize}

\begin{itemize}
\item \texttt{\textasciicircum{}\textasciicircum{}} LaTeX's special syntax for characters results in \textquotedbl{}strange\textquotedbl{} replacements
(both with \texttt{\textbackslash{}href} and \texttt{\textbackslash{}url}). The writer emits a warning.

\href{../strange^^name}{file with \textasciicircum{}\textasciicircum{}}:
\url{../strange^^name}

\item Unbalanced braces, \{ or \}, will fail (both with \texttt{\textbackslash{}href} and \texttt{\textbackslash{}url}):

\begin{quote}
\begin{alltt}
`file with \{ <../strange\{name>`__
`<../strange\{name>`__
\end{alltt}
\end{quote}

while balanced braces are suported:

\begin{DUlineblock}{0em}
\item[] \url{../strange{n}ame}
\item[] \url{../st{r}ange{n}ame}
\item[] \url{../{st{r}ange{n}ame}}
\end{DUlineblock}
\end{itemize}

\end{document}
